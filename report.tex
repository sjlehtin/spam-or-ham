\documentclass[a4paper,10pt]{article}

\usepackage[english]{babel}
\usepackage[utf8]{inputenc}
\usepackage{graphicx}
\usepackage[tbtags]{amsmath}
\usepackage{amssymb}
\usepackage{amsbsy}
\usepackage{eucal}
\usepackage{hyperref, url}
\usepackage{float}
\usepackage{verbatim} % For verbatiminput.
\usepackage{fancyvrb}

\pdfinfo{            
          /Title      (T-61.3050 Machine Learning: Basic Principles)
          /Author     ()
          /Keywords   ()
}

\DefineVerbatimEnvironment {code}{Verbatim}
   {%numbers=left,numbersep=2mm,
     %frame=lines,framerule=0.1mm, 
     fontsize=\small}
\RecustomVerbatimCommand{\VerbatimInput}{VerbatimInput}
   {%numbers=left,numbersep=2mm,
     frame=lines,framerule=0.1mm, fontsize=\small}
\DefineVerbatimEnvironment {outputlog}{Verbatim}
  {frame=lines,framerule=0.1mm, fontsize=\small}

\newcommand{\XXX}[1]{{\bf XXX #1}}
\parindent 0mm
\parskip 3mm

%------------------------------------------------------------------------------
% First page
%------------------------------------------------------------------------------

% add your student number in parenthesis
\title{T-61.3050 Term Project, final report\\ % Dictated in instructions.
       Using decision trees for spam classification}
\author{Jori Bomanson (81819F) \\
  {\tt jori.bomanson@aalto.fi} \\
  \\
  Sami J. Lehtinen (44814P)\\ 
  {\tt sjl@iki.fi} \\
}
\begin{document}

\floatstyle{plain}
\newfloat{Listing}{t}{lol}
\floatname{Listing}{Listing}

\maketitle
\thispagestyle{empty}
\pagebreak
\pagenumbering{arabic}

\section{Abstract}
% Include the key points of your work in a few sentences (including methods
% used and conclusions).
We trained a univariate binary classification tree to detect spam email.


\section{Rationale}

The choice of using a decision tree to solve a task with boolean variables
seemed natural. Such a tree could definitely match whatever complexity there
would be in the training data.
In fact we figured our success would be determined mainly by how we would
manage to avoid overfitting.

Decisiontrees were also tempting in that they are much easier to
visualize than, e.g., naive bayes classifiers.  For example, see
\XXX{references to figures}

\section{Principles of decision trees}

Decision trees are a non-parametric method for data classification.
They don't try to approximate any model behind the data, instead relying
on tactic of ``similar inputs produce similar outputs''.

\subsection{Calculating impurity}

We chose to measure impurity with the entropy function \XXX{proper
citing} (Quinlan 1986)
\begin{equation*}
\begin{split}
\mathcal{I}_m &= - \sum_{i=1}^K p_m^i \log_2 p_m^i  \\
\text{for a node} \quad & m  \\
\text{where} \quad 0 \log 0 &\equiv 0  \\
p_m^i &= P(\text{Instance reaching node m belongs to class i})  \\
K &= 2
\end{split}
\end{equation*}

\subsection{Pruning the tree}

\subsubsection{Prepruning}

In our first versions of the classifier we only did prepruning, which
resulted in very small and tidy trees, but the accuracy left a lot to be
desired. \XXX{some data} Current model uses one of the suggested
approaches from \cite{alpaydin2004}, where we make the tree as pure as
possible (no prepruning) and handle overfitting by postpruning the tree.
Removing prepruning didn't have an adverse effect for the final
accuracy. \XXX{data}

When prepruning, we stopped the algorithm by limiting by the results
from the entropy calculation, by columns (features) and rows (messages)
left in a recursion step.  For example, \XXX{more detailed explanation}.

\subsubsection{Postpruning}
\label{sect:postpruning}



\section{Validation of our approach}

\subsection{K-fold cross validation}

We implemented 10-fold cross validation to our classifier.  This
gave more stable values for generalization error when compared to
repeatedly training the classifier without the cross validation.

One part of the training data set was used for postpruning the tree, see
section \ref{sect:postpruning}.

\subsection{Dummy model}


\section{Results with different training set sizes}

XXX results to a table.

1000 items.

\begin{outputlog}
Chosen classifier has accuracy of 1.0.
Classified as spam: 5777 / 9000
Validation:
 Classified as spam: 5777 / 9000
 Classified as spam in validation set: 6134 / 9000
 Correctly classified: 8475
 Correctly classified as spam: 5693
 Accuracy: 0.942
 Precision: 0.985
 Recall: 0.928

real	0m44.221s
user	0m44.040s
sys	0m0.150s
\end{outputlog}

4000 items in training set.

\begin{verbatim}
Chosen classifier has accuracy of 0.9725.
Classified as spam: 3984 / 6000
Validation:
 Classified as spam: 3984 / 6000
 Classified as spam in validation set: 4064 / 6000
 Correctly classified: 5798
 Correctly classified as spam: 3923
 Accuracy: 0.966
 Precision: 0.985
 Recall: 0.965

real	9m41.857s
user	9m40.900s
sys	0m0.520s
\end{verbatim}

6000 items in training set.

\begin{verbatim}
Chosen classifier has accuracy of 0.978333333333.
Classified as spam: 2677 / 4000
Validation:
 Classified as spam: 2677 / 4000
 Classified as spam in validation set: 2725 / 4000
 Correctly classified: 3886
 Correctly classified as spam: 2644
 Accuracy: 0.972
 Precision: 0.988
 Recall: 0.970

real	19m40.578s
user	19m37.980s
sys	0m1.100s
\end{verbatim}

\section{Conclusions}

The chosen approach didn't achieve the same high marks for accuracy as
the winners of the data challenge.  Tweaking of the approach is
naturally possible, but it seems that decisiontrees are more suited in
this as a part of combined classifier instead of on their own.  This
does not mean that the approach is completely without merits; decision
trees were applied here without prior knowledge to the distribution of
the spam messages.  In the data challenge, the winning teams had tweaked
the prior probabilities for spam to match the distribution in the data
set.  In the decision tree model, this is not not necessary, or even
possible.

Looking at the generated decision trees, it is easy to follow the
process by eye.  This, we think, helps a lot in validating the approach.

\section{Comments on project difficulty}

\begin{thebibliography}{9}
\bibitem{alpaydin2004}
  Ethem Alpaydin,
  \emph{Introduction to Machine Learning}.
  Massachusetts Institute of Technology, Cambridge, Massachusetts,
  1st edition,
  2004. 415 pages. ISBN 0-262-01211-1.
%\bibitem{press07}
%  William H. Press, Saul A. Teukolsky, William T. Vetterling, Brian P. Flannery,
%  \emph{Numerical recipes - the art of scientific computing}.
%  Cambridge University Press, New York,
%  3rd edition,
%  2007. 1235 pages. ISBN 978-0-521-88068-8.
%\bibitem{mellin10a}
%  Ilkka Mellin, 
%  \emph{Todennäköisyyslaskenta ja tilastotiede: Kaavat}.
%  Otaniemi, 2010. 390 pages.
%\bibitem{mellin10b}
%  Ilkka Mellin, 
%  \emph{Tilastolliset taulukot}.
%  Otaniemi, 2010. 13 pages.
\end{thebibliography}

\appendix
\section{Python code for the classifier}
\verbatiminput{decisiontree.py}

\end{document}
